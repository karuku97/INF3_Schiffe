\section{Einführung}

Die Aufgabe für dieses Projekt war es, ein Programm zu erstellen, mit dessen Hilfe sich eine bzw. mehrere Strategien für das ''Spiel Einbahn Schiffe versenken'' bewerten ließ. Dafür 
sollte 
ein Server implementiert werden, welcher ein Spielfeld erzeugt,\\ Koordinaten entgegennimmt und entsprechend dem Status des angefragten Feldes eine passende Antwort zurückgibt (siehe 3.1 Protokoll Server). 
Des Weiteren sollte ein Client implementiert werden.
Dieser soll eine eigenständig programmierte Strategie \\
automatisch nach einer Aufforderung und ohne weitere Eingaben eines Benutzers ausführen, bis das Spiel beendet ist. \\
Eine weitere Funktion des Clients soll sein, dass dieser die benötigten Spielzüge zählt, speichert und am Ende ausgibt. Ebenfalls sollen die 
Strategien auf ihre mittlere Anzahl von Spielzügen verglichen werden.\newline

Das Projekt startete Ende Oktober 2021 und am 04.11.2021 wurden erste Dateien im Repository hochgeladen.\newline
Dabei handelte es sich um die Readme und Dateien die uns zur Verfügung gestellt wurden.
Dazu gehörten: TASK3, Client, Server, Simplesocket und ein Makefile.\newline
\newline
Die ersten Strategien wurden am 17.11.21 hochgeladen, einmal Intellistrat und \newline IntellistratDiagonal, sowie Randshoot.
Die Funktion Randshoot war zu diesem Zeitpunkt noch als Funktion in der Client Datei und wurde erst später Ausgelagert.\newline
\newline
Ende November kamen noch die Strategien Bruteforce und BruteforceDiagonal dazu und die Funktion Randshoot wurde als eigenständige Datei ausgelagert.\newline
\newline
Im Dezember wurde noch die Funktion RandshootIs hinzugefügt. Es fanden ebenfals größere Überarbeitungen des Codes statt, unteranderem die Aufteilung der Strategien in Header- und C++ 
-Files.\newline Ende Dezember war die erste Version des Projekts fertig, sodass im Januar 2022 nur kleinere Änderungen des Codes und Bugfixes nötig waren und eine erste statistische 
Auswertung zur durchschnittlichen Anzahl von Schüssen der einzelnen Strategien möglich war. 

\subsection{Starten des Programms}
Zum Starten des Programms wird erst der Befehl make all in die Konsole eingegeben. Danach wird in einer separaten Konsole erst der Server mit dem Befehl "./Server" gestartet. In der 
anderen Konsole wird dann der Client gestartet, Dies geschied durch die Eingabe des Befehls "./Client 20 3" die beiden Zahlen sind beispielhaft, die erste Zahl steht für die Anzahl 
der Spiele die gespielt werden sollen und die zweite Zahl steht Stellvertretend für die Methode die für die Spiele verwendet werden soll.

\begin{table}[h]
    \begin{tabular}{lllll}
     Strategienummer & Name &  &  &  \\
     0 & BRUTEFORCE &  &  &  \\
     1 & BRUTEFORCEDIAGONAL &  &  &  \\
     2 & RANDSHOOT &  &  &  \\
     3 & RANDSHOOTIS &  &  &  \\
     4 & INTELLISTRAT &  &  &  \\ 
     5 & INTELLISTRATDIAGONAL &  &  & 
    \end{tabular}
    \end{table}

Nachdem Ende jedes Spiel wird die Anzahl der Schüsse in der Konsole ausgegeben und kann zur Auswertung verwendet 
werden.


