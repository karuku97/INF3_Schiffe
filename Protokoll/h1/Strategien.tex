\section*{Strategien und Kommunikation}

\subsection*{Client}




\subsection*{Server}

\subsection*{Protokoll Server}
Zur Kommunikation zwischen Server und Client läuft nach folgendem Protokoll ab.\\
\includegraphics[]{Screenshot_2022-01-12 Commits · karuku97 INF3_Schiffe.png}
\\
\subsubsection*{Strategien}
Zum beschießen des erzeugten Spielfelds werden (hier finale Anzahl einfügen) unterschiedliche Strategien verwendet.

\subsubsection*{Bruteforce}

\subsubsection*{BruteforceDiagonal}

\subsubsection*{Randshoot}

\subsubsection*{RandshootIs}

\subsubsection*{Intellistrat}

\subsubsection*{IntellistratDiagonal}

\subsection*{Spielfeldverwaltung}

Die Spielfeldverwaltung setzt sich aus folgenden Funktionen und einer Klasse zusammen.
Die gleichnahmige Klasse SpielfeldVerwaltung, besitz ein eindimensionales Array Spielfeld mit 100 Variablen des Typs Enum(die dazugehörigen Enums sind ausserhalb der Klasse definiert)
, die integer Variablen lastX, lastY, lastPos und die Funktionen CoordsToPosition, int SpielfeldPositionToCoordsX, SpielfeldPositionToCoordsY, SchiffePositionToCoordsX, 
SchiffePositionToCoordsY, getFieldstatus, getLastFieldStatus, Statusreport, SchiffPosition, ServerStringToEnum. Die Funktionen CoordsToPosition, int SpielfeldPositionToCoordsX, 
SpielfeldPositionToCoordsY, SchiffePositionToCoordsX, SchiffePositionToCoordsY, getFieldstatus, searchShipclass leifern einen Rückgabewert des Typs Integer, Statusreport, 
SchiffPosition liefern keinen Rückgabewert und Serverstring toEnum liefert einen Enum als Rückgabewert. Das eindimensionale Array Spielfeld ist dafür da um den Status des 
beschossenen Feldes zu Speichern, folgende Möglichkeiten bestehen NICHT\_BESCHOSSEN = 0, WASSER = 1, SCHIFF\_GETROFFEN = 2, SCHIFF\_ZERSTOERT = 3, GAMEOVER = 4, ERROR = -1


Erklärung der einzelnen Strategien, was unterscheidet die Ideen
Protokoll Server Client Kommunikation